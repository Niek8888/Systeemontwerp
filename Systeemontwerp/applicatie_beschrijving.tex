\newpage
\section{Applicatie beschrijving}
Voor het vak Systeemontwerp is het doel om energie zuinig elektronica te ontwikkelen. Dit willen doen door het maken van een sensor die de absolute en relatieve luchtvochtigheid, evenals de temperatuur, kan meten in een ruimte. Dit systeem is bedoeld om de temperatuur en luchtvochtigheid in kantoorpanden in kaart te brengen, zodat de werkomgeving geoptimaliseerd kan worden. Hierdoor kan efficiënter gebruik worden gemaakt van de airconditioning- en ventilatiesystemen. Om dit te bereiken maken we gebruik van twee sensoren: een temperatuursensor en een luchtvochtigheidssensor.
\section{Theoretisch kader}
\label{Theoretisch_kader}
Volgens de arbo is de maximaal Temperatuur in kantoor waar licht fysiek werk wordt verricht 28 $^\circ\text{C}$ \cite{ARBO_temperatuur_regels}. voor zwaar fysiek werk is dit zelfs strenger en mag de maximale temperatuur niet boven de 23 $^\circ\text{C}$ uitkomen. Dit zijn wettelijke grenswaarden maar eerder een indicatie. Om ervoor te zorgen dat de sensor boven dit bereik ook nog blijft werken is een marge van besloten om marge van 17 $^\circ\text{C}$ aan te houden. Dit resulteert in een maximale meet temperatuur van 55 $^\circ\text{C}$. In de Arbo staat geen minimale temperatuur voor een kantoor ruimte. Hierdoor is er besloten om de sensor te laten met tot -5 $^\circ\text{C}$. Dit zorgt voor een meet bereik van 60 $^\circ\text{C}$. Voor dit meetbereik is besloten om een temperatuur accurate van 0.1 $^\circ\text{C}$. Dit gedaan om de airconditioning- en ventilatiesystemen zo accurate mogelijk informatie te geven en hiermee efficiënter de kantoorruimte te reguleren qua temperatuur
\newline
\\
De Beste werk omstandigheden voor een werknemer op kantoor liggen rond de 22 $^\circ\text{C}$ \cite{Beste_werk_omstandigheden}. Dit is in samen hang met een relatieve luchtvochtigheid van 50 procent. Om dit goed in kaart te brengen is een relatieve luchtvochtigheidssensor nodig met een resolutie van 0 tot 100 procent. Omdat de productiviteit van werknemers af neemt bij een te hoge temperatuur of luchtvochtigheid. Om ervoor te zorgen dat werknemers het beste kunnen presteren en een prettige werkomgeving hebben is het nodig om dit in kaart te brengen voor elke werkplek afzonderlijk. Hierdoor is het mogelijk om een kaart van de werkomgeving van alle werknemers te maken. Door deze informatie te verwerken en hiermee de airconditioning- en ventilatiesystemen nauwkeurig aan te sturen is het mogelijk om de werknemers op hun hoogste productiviteit te houden.