\newpage
\section{Applicatie beschrijving}
Voor het vak Systeemontwerp is het doel om energie zuinig elektronica te ontwikkelen. Hierom is er voor gekozen om een sensor te realiseren die de absolute en relatieve luchtvochtigheid, evenals de temperatuur kan meten in een ruimte. Dit systeem is bedoeld om de temperatuur en luchtvochtigheid in kantoorpanden in kaart te brengen, zodat de werkomgeving geoptimaliseerd kan worden. Hierdoor kan efficiënter gebruik worden gemaakt van de airconditioning- en ventilatiesystemen. Om dit te bereiken maken we gebruik van twee sensoren: een temperatuursensor en een luchtvochtigheidssensor.
\section{Theoretisch kader}
\label{Theoretisch_kader}
Volgens de arbo is de maximaal Temperatuur in kantoor waar licht fysiek werk wordt verricht 28 $^\circ\text{C}$ \cite{ARBO_temperatuur_regels}. voor zwaar fysiek werk is dit zelfs strenger en mag de maximale temperatuur niet boven de 23 $^\circ\text{C}$ uitkomen. Dit zijn geen wettelijke grenswaarden, maar eerder een indicatie. Om ervoor te zorgen dat de sensor boven dit bereik ook nog blijft werken is een marge van besloten van 17 $^\circ\text{C}$ boven licht fysiek werk aan te houden. Dit resulteert in een maximale meet temperatuur van 55 $^\circ\text{C}$. In de Arbo staat geen minimale temperatuur voor een kantoorruimte. Hierom is er besloten om de laagste temperatuur die de sensor kan meten op -5 $^\circ\text{C}$ te zetten. Dit zorgt voor een meetbereik van 60 $^\circ\text{C}$. Voor dit meetbereik is besloten om een temperatuur accurate van 0.1 $^\circ\text{C}$. Dit gedaan om de airconditioning- en ventilatiesystemen zo accurate mogelijk informatie te geven en hiermee efficiënter de kantoorruimte te reguleren qua temperatuur
\newline
\\
De beste werkomstandigheden voor een werknemer op kantoor liggen rond de 22 $^\circ\text{C}$ \cite{Beste_werk_omstandigheden}. Dit is in samenhang met een relatieve luchtvochtigheid van 50 procent. Om de werkomstandigheden goed in te brengen van een kantoor is een relatieve luchtvochtigheidssensor nodig met een resolutie van 0 tot 100 procent. Omdat de productiviteit van werknemers neemt af bij een te hoge temperatuur of luchtvochtigheid. Om ervoor te zorgen dat werknemers het beste kunnen presteren en een prettige werkomgeving hebben is het nodig om de werkomstandigheden in kaart te brengen voor elke werkplek afzonderlijk. Hierdoor is het mogelijk om een kaart van de werkomgeving van alle werknemers te maken. Door deze informatie te verwerken en hiermee de airconditioning- en ventilatiesystemen nauwkeurig aan te sturen is het mogelijk om de werknemers op hun hoogste productiviteit te houden.
\newline
\\
airconditioning- en ventilatiesystemen werken meestal door middel van cascade controle systemen aangestuurd. Deze systemen hebben een regeltijd van tientallen seconden tot minuten \cite{Wang2008}. Om ervoor te zorgen dat onze applicatie kan werken met deze systemen is het nodig dat er snel genoeg data naar gestuurd kan worden. Hierom is er gekozen voor een update frequentie van minimaal 0,1 Hz.